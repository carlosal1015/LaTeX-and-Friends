%!TEX root = presentazionelancia.tex
\section{Workload \& Performance}
%\begin{frame}[c]\frametitle{Workload}
%\centering
%\includegraphics[width=\textwidth]{figs/table3.pdf}
%
%Frequencies for client request
%
%
%
%\end{frame}

%\begin{frame}[c]\frametitle{Disponibilidad}
%Bajo carga de trabajo real, durante un período de 90 días, la \textbf{fracción de consultas TAO fallidas} es:
%\begin{center}
%	\huge $4.9 \times 10^{-6}$
%\end{center}
%
%\end{frame}
%\begin{frame}[c]\frametitle{Followers Capacity}
%	\centering
%    \includegraphics[width=\textwidth]{figs/followercapacity.pdf}
%\end{frame}

%\begin{frame}[c]\frametitle{Hit Rates and read latency}
%\centering
%\includegraphics[width=\textwidth]{figs/table8.pdf} 
%
%\end{frame}

%\begin{frame}[c]\frametitle{Write Latency}
%   	\includegraphics[width=\textwidth]{figs/writes.pdf}
%\end{frame}


\begin{frame}[c]\frametitle{Resumiendo}
\begin{description}
	\item[Instalación] \hfill\\
	\begin{itemize}
		\item \TeX{} Live 2018
		\item OS GNU/Linux
		\item Sage 8.4
	\end{itemize}
	\item[Estudio]\hfill\\
	\begin{itemize}
		\item Teoría de los libros
		\item Práctica con ejercicios de ejemplos
	\end{itemize}
	\item[Comunicación y documentación]\hfill\\
	\begin{itemize}
		\item A través de Gitter o Hangout + GitHub.
	\end{itemize}
\end{description}


\end{frame}