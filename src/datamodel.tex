%!TEX root = presentazionelancia.tex
\section{Objetivos}

\begin{frame}
\frametitle{\insertsection}
 	Queremos formar un grupo humano en la UNI alrededor de \LaTeX{} tal como lo hace el Grupo de Usuarios \TeX{} de la Universidad de Tokio.
 	
 	\url{http://ut-tex.org/index.php}
 	\begin{center}
 	\includegraphics[width=\textwidth]{figs/tokio}		
 	\end{center}
\end{frame}

\begin{frame}[fragile]
\frametitle{¿Cómo lo podemos lograr?}
\onslide<1->
\begin{itemize}
	\item Estudiar los libros de Herbert Voss, vicepresidente del Grupo de Usuarios \TeX{} Germano hablantes.
	\item Conocer herramientas amigas de \LaTeX{} como Sage, arara, PSTricks, knitr y más tecnologías. 
	\item Con el uso de un sistema de control de versiones descentralizado como \texttt{git}
	\item Herramientas de comunicación como \texttt{gitter} o \texttt{Hangouts}.
\end{itemize}
\onslide<2->
Conocimientos previos:
\begin{itemize}
	\item Comandos básicos en \texttt{UNIX}, \LaTeX{} y programación.
\end{itemize}
\end{frame}


%\begin{frame}[fragile]
%\frametitle{Associations}
%    \begin{itemize}
%	\item Typed directed edges between objects (type is denoted by \verb!atype!)
%	\item Identified by source object \verb!id1!, \verb!atype! and destination object \verb!id2!
%	\item Contains data in the form of key-value pairs.
%	\item Contains a 32-bit \verb!time! field.
%	\item Models actions that happen at most once or records state transition (e.g. like)
%	\item Often inverse association is also meaningful (eg like and liked by).
%\end{itemize}
%\end{frame}

%\begin{frame}
%\frametitle{Associations API}
%\begin{itemize}
%\item Add new
%\item Delete
%\item Change type
%\end{itemize}
%Also inverse association is created or modified automatically
%\end{frame}

%\begin{frame}[fragile]
%\frametitle{Querying TAO}
%TAO's associations queries are organized around \emph{associations~ list}
%
%\begin{itemize}
%\item \verb!assoc_get(id1,atype, id2set, high?, low?)!
%\item \verb!assoc_count(id1,atype)!
%\item \verb!assoc_range(id1, atype, pos, limit)!
%\item \verb!assoc_time_range(id1,atype, high, low, limit)!
%\end{itemize}
%
%Query results are bounded to 6000 results
%\end{frame}



