% arara: xelatex
% arara: clean: {extensions: ['log','out','snm']}
\documentclass[spanish]{beamer}
\usepackage{euler}
\usepackage{graphicx,hyperref,url, materialbeamer}
\usepackage{braket}
\usepackage{listings}


\graphicspath{ {./figs/} }
\setbeamercovered{transparent}
\lstdefinestyle{customsql}{
  belowcaptionskip=1\baselineskip,
  breaklines=true,
  xleftmargin=\parindent,
  language=SQL,
  showstringspaces=false,
  basicstyle=\footnotesize\ttfamily,
  keywordstyle=\bfseries\color{green!40!black},
  commentstyle=\itshape\color{purple!40!black},
  identifierstyle=\color{blue},
  stringstyle=\color{orange},
}
\lstset{escapechar=@,style=customsql}



\usefonttheme{professionalfonts} % using non standard fonts for beamer
%\usefonttheme{serif}

% The title of the presentation:
%  - first a short version which is visible at the bottom of each slide;
%  - second the full title shown on the title slide;
\title[\LaTeX{} \& Friends]{\LaTeX{} \& friends}

% Optional: a subtitle to be dispalyed on the title slide
\subtitle{SageMath, arara, PSTricks \& knitr\\[\baselineskip]
20 de octubre del 2018}

% The author(s) of the presentation:
%  - again first a short version to be displayed at the bottom;
%  - next the full list of authors, which may include contact information;
\author[Dimension R]{Carlos Aznarán} 
  
%\titlegraphic{\includegraphics[width=\textwidth]{atac-logo}}

% The institute:
%  - to start the name of the university as displayed on the top of each slide
%    this can be adjusted such that you can also create a Dutch version
%  - next the institute information as displayed on the title slide
\institute[Sapienza Università di Roma]{
Matemáticas\\
  Facultad de Ciencias\\
  Universidad Nacional de Ingeniería}

% Add a date and possibly the name of the event to the slides
%  - again first a short version to be shown at the bottom of each slide
%  - second the full date and event name for the title slide
\date[\today]{
 \today}

\providecommand{\di}{\mathop{}\!\mathrm{d}}
\providecommand*{\der}[3][]{\frac{d\if?#1?\else^{#1}\fi#2}{d #3\if?#1?\else^{#1}\fi}} 
 \providecommand*{\pder}[3][]{% 
    \frac{\partial\if?#1?\else^{#1}\fi#2}{\partial #3\if?#1?\else^{#1}\fi}% 
  }
\begin{document}

\begin{frame}
  \titlepage
\end{frame}

\begin{frame}
  \frametitle{\contentsname}

  \tableofcontents
\end{frame}

%!TEX root = presentazionelancia.tex

\setlength{\parskip}{\baselineskip} 
\section{Primeros pasos}
\begin{frame}[t]
\frametitle{\insertsection}
\begin{block}{¿Qué es \LaTeX{} \& friends?}
	Es un taller tiene como metas
	\begin{itemize}
		\item duración hasta fin de ciclo.
	 	\item resolución de los ejercicios de los libros
	 	\item embeber y extender \TeX{} a Python, plataformas web (MathJax, Ka\TeX, MathML) y plataformas de escritorio.
	 	\item presentar informes matemáticos de alta calidad.
	 \end{itemize} 
\end{block}
\end{frame}

%\begin{frame}
%\frametitle{The social graph}
%Facebook has more than 1 billion active user 
%\begin{itemize}
%	\item recording relationships,
%	\item sharing interests,
%	\item uploading pictures and \dots
%\end{itemize}
%
%The user experience of Fb comes from rapid, efficient and scalable access to the \emph{social graph}
%\end{frame}
%
%\begin{frame}
%	What's behind an entry in yours Fb page?
%
%	A single Fb page aggregate and filter hundreds of items from the social graph.
%\end{frame}%

%\begin{frame}[t]
%\begin{center}
%\includegraphics[width=\paperheight]{figs/gitter}	
%\end{center}


%\end{frame}
%%%
{
	\usebackgroundtemplate{\centering\includegraphics[width=\paperwidth]{gitter}}
	\begin{frame}[plain]
\end{frame}
}
%%%%
%\begin{frame}
%\frametitle{Before Tao}
%	\begin{itemize}
% 	\item Facebook was storing the social graph to MySql
%	\begin{itemize}
%		\item  	Quering it from PHP
%		\item  	Storing result in memcache\\
%	\end{itemize}
%	\end{itemize}
%	\begin{center}
%		\includegraphics[width=0.3\textwidth]{figs/php-logo.eps}\quad
%		\includegraphics[width=0.3\textwidth]{figs/mysql.png}
%	\end{center}
% 	Over time Fb deprecated direct access to MySQL in favor of a graph (associations, nodes) abstraction
%\end{frame}

%\begin{frame}
%\frametitle{Limits}
%    \begin{itemize}
%    	\item Operations on lists are inefficient in memcache (update whole list)
%    	\item Complexity on clients managing cache
%    	\item Hard to offer read-after-write consistency
%    \end{itemize}
%Also they want to access social graph from non-PHP services
%\end{frame}
%
%\begin{frame}
%\frametitle{TAO's Goals}
%	\begin{itemize}
%		\item Efficiency at Scale
%		\pause
%		\item Low read latency
%		\pause
%		\item Timeliness of writes
%		\pause
%		\item High read availability
%	\end{itemize}
%\end{frame}

%!TEX root = presentazionelancia.tex
\section{Objetivos}

\begin{frame}
\frametitle{\insertsection}
 	Queremos formar un grupo humano en la UNI alrededor de \LaTeX{} tal como lo hace el Grupo de Usuarios \TeX{} de la Universidad de Tokio.
 	
 	\url{http://ut-tex.org/index.php}
 	\begin{center}
 	\includegraphics[width=\textwidth]{figs/tokio}		
 	\end{center}
\end{frame}

\begin{frame}[fragile]
\frametitle{¿Cómo lo podemos lograr?}
\onslide<1->
\begin{itemize}
	\item Estudiar los libros de Herbert Voss, vicepresidente del Grupo de Usuarios \TeX{} Germano hablantes.
	\item Conocer herramientas amigas de \LaTeX{} como Sage, arara, PSTricks, knitr y más tecnologías. 
	\item Con el uso de un sistema de control de versiones descentralizado como \texttt{git}
	\item Herramientas de comunicación como \texttt{gitter} o \texttt{Hangouts}.
\end{itemize}
\onslide<2->
Conocimientos previos:
\begin{itemize}
	\item Comandos básicos en \texttt{UNIX}, \LaTeX{} y programación.
\end{itemize}
\end{frame}


%\begin{frame}[fragile]
%\frametitle{Associations}
%    \begin{itemize}
%	\item Typed directed edges between objects (type is denoted by \verb!atype!)
%	\item Identified by source object \verb!id1!, \verb!atype! and destination object \verb!id2!
%	\item Contains data in the form of key-value pairs.
%	\item Contains a 32-bit \verb!time! field.
%	\item Models actions that happen at most once or records state transition (e.g. like)
%	\item Often inverse association is also meaningful (eg like and liked by).
%\end{itemize}
%\end{frame}

%\begin{frame}
%\frametitle{Associations API}
%\begin{itemize}
%\item Add new
%\item Delete
%\item Change type
%\end{itemize}
%Also inverse association is created or modified automatically
%\end{frame}

%\begin{frame}[fragile]
%\frametitle{Querying TAO}
%TAO's associations queries are organized around \emph{associations~ list}
%
%\begin{itemize}
%\item \verb!assoc_get(id1,atype, id2set, high?, low?)!
%\item \verb!assoc_count(id1,atype)!
%\item \verb!assoc_range(id1, atype, pos, limit)!
%\item \verb!assoc_time_range(id1,atype, high, low, limit)!
%\end{itemize}
%
%Query results are bounded to 6000 results
%\end{frame}




\input{architechture}
\input{implementation}
\input{consistency}
%!TEX root = presentazionelancia.tex
\section{Workload \& Performance}
%\begin{frame}[c]\frametitle{Workload}
%\centering
%\includegraphics[width=\textwidth]{figs/table3.pdf}
%
%Frequencies for client request
%
%
%
%\end{frame}

%\begin{frame}[c]\frametitle{Disponibilidad}
%Bajo carga de trabajo real, durante un período de 90 días, la \textbf{fracción de consultas TAO fallidas} es:
%\begin{center}
%	\huge $4.9 \times 10^{-6}$
%\end{center}
%
%\end{frame}
%\begin{frame}[c]\frametitle{Followers Capacity}
%	\centering
%    \includegraphics[width=\textwidth]{figs/followercapacity.pdf}
%\end{frame}

%\begin{frame}[c]\frametitle{Hit Rates and read latency}
%\centering
%\includegraphics[width=\textwidth]{figs/table8.pdf} 
%
%\end{frame}

%\begin{frame}[c]\frametitle{Write Latency}
%   	\includegraphics[width=\textwidth]{figs/writes.pdf}
%\end{frame}


\begin{frame}[c]\frametitle{Resumiendo}
\begin{description}
	\item[Instalación] \hfill\\
	\begin{itemize}
		\item \TeX{} Live 2018
		\item OS GNU/Linux
		\item Sage 8.4
	\end{itemize}
	\item[Estudio]\hfill\\
	\begin{itemize}
		\item Teoría de los libros
		\item Práctica con ejercicios de ejemplos
	\end{itemize}
	\item[Comunicación y documentación]\hfill\\
	\begin{itemize}
		\item A través de Gitter o Hangout + GitHub.
	\end{itemize}
\end{description}


\end{frame}

\setbeamercolor{background canvas}{bg=matbluedark}
\setbeamercolor{normal text}{fg=white}
\begin{frame}[plain, b]
\centering
\huge \textcolor{white}{¡Gracias!}
\normalsize
\begin{center}
	\includegraphics[width=0.5\textwidth]{figs/Dimensión_R_Logo}	\\
\end{center}
\vspace*{\fill}

 \begin{beamercolorbox}[wd=\paperwidth]{section in head/foot}
 \centering\large
\LaTeX{} \& friends -- con el soporte de Dimension R
\vskip10pt
\end{beamercolorbox}
 \end{frame}

\end{document}
