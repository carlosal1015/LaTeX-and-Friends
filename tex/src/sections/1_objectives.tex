\section{Objetivos}

\begin{frame}
\frametitle{\insertsection}
Queremos formar un grupo humano en la UNI alrededor de \LaTeX{} tal como lo hace el Grupo de Usuarios \TeX{} de la Universidad de Tokio.

{\centering
\url{http://ut-tex.org/index.php}
\par}

	\begin{center}
	\includegraphics[width=\textwidth]{tokio}
	\end{center}
\end{frame}

\begin{frame}[fragile]
\frametitle{¿Cómo lo podemos lograr?}
\onslide<1->
\begin{itemize}
	\item Estudiar los libros de Herbert Vo\ss, vicepresidente del Grupo de Usuarios \TeX{} Germano hablantes.
	\item Conocer herramientas amigas de \LaTeX{}, tales como Sage, \arara, PSTricks, knitr y más tecnologías.
	\item Emplear un sistema de control de versiones descentralizado como \texttt{git}.
	\item Herramientas de comunicación como \href{https://gitter.im/}{\texttt{gitter}} o \href{https://hangouts.google.com/}{\texttt{Hangouts}}.
\end{itemize}
\onslide<2->
\textbf{Conocimientos previos:}
\begin{itemize}
	\item Comandos básicos en \texttt{UNIX}, \LaTeX{} y programación.
\end{itemize}
\end{frame}