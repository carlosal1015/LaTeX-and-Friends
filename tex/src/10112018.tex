% arara: xelatex
% arara: sage
% arara: xelatex
% arara: clean: {extensions: ['log','out','snm','listing','nav','sagetex.sage','sagetex.sage.py','sagetex.scmd','sagetex.sout','toc','vrb']}
\documentclass[spanish,9pt]{beamer}
\usepackage{euler}
\usepackage{graphicx,hyperref,url, materialbeamer}
\usepackage{braket}
\usepackage{listings}
\usepackage{sagetex}

\graphicspath{ {../img/} }
\setbeamercovered{transparent}

\usefonttheme{professionalfonts} % using non standard fonts for beamer

\title[\LaTeX{} \& Friends]{\LaTeX{} \& friends}

% Optional: a subtitle to be dispalyed on the title slide
\subtitle{SageMath, arara, PSTricks \& knitr\\[\baselineskip]
10 de noviembre del 2018}

% The author(s) of the presentation:
%  - again first a short version to be displayed at the bottom;
%  - next the full list of authors, which may include contact information;
\author[Dimension R]{Carlos Aznarán} 

%\titlegraphic{\includegraphics[width=\textwidth]{usa}}

\institute[Universidad Nacional de Ingeniería]{
Matemáticas\\
  Facultad de Ciencias\\
  Universidad Nacional de Ingeniería}

\date[\today]{
 \today}

\providecommand{\di}{\mathop{}\!\mathrm{d}}
\providecommand*{\der}[3][]{\frac{d\if?#1?\else^{#1}\fi#2}{d #3\if?#1?\else^{#1}\fi}} 
 \providecommand*{\pder}[3][]{% 
    \frac{\partial\if?#1?\else^{#1}\fi#2}{\partial #3\if?#1?\else^{#1}\fi}% 
  }
\renewcommand{\sin}{\mathrm{sen}}
\begin{document}

\begin{frame}
  \titlepage
\end{frame}

\begin{frame}
  \frametitle{\contentsname}

  \tableofcontents
\end{frame}

\section{Background}
\begin{frame}[fragile]
\frametitle{\insertsection}
\end{frame}

\section[\arara]{\arara}

\subsection{¿Qué es esta herramienta?}

\begin{frame}
\frametitle{\insertsection}
\begin{block}{\insertsubsection}
	\begin{itemize}
		\item Herramienta de automatización \TeX{} basada en reglas y directivas.
		\item Control de los documentos: \arara\ no hará algo a menos que le enseñes la tarea y le digas explícitamente la tarea a ejecutar.
	\end{itemize}
\end{block}
\end{frame}

\subsection{Conceptos claves}

\begin{frame}
\frametitle{\insertsection}
\begin{block}{\insertsubsection}
\begin{itemize}
	\item Reglas: Descripción formal de cómo \arara\ maneja una determinada tarea.
	\item Directivas: Comentario especial que se inserta en el código fuente en el que le indicas cómo \arara\ debería comportarse.
	\item Ejemplos de directivas: \texttt{latex}, \texttt{xelatex}, \texttt{luatex}, \texttt{clean}, \texttt{indent}. \texttt{make}, \texttt{xindy}, \texttt{makeglossaries}, incluso puedes crear tus propias directivas.
\end{itemize}
\end{block}
\end{frame}

\subsection{Algunos métodos}

\begin{frame}
\frametitle{\insertsection}
\begin{block}{\insertsubsection}

\end{block}
\end{frame}

\subsection{Cajas de diálogo}

\begin{frame}
\frametitle{\insertsection}
\begin{block}{\insertsubsection}
Es un elemento de control gráfico, generalmente una pequeña ventana, que comunica información al usuario y le solicita una respuesta.
\end{block}
\end{frame}

\begin{frame}[fragile]
\frametitle{\insertsection}
\begin{codebox}{Terminal}{teal}{\icnote}{white}
$ arara hello.tex 
__ _ _ __ __ _ _ __ __ _ 
/ _` | '__/ _` | '__/ _` |
| (_| | | | (_| | | | (_| |
\__,_|_|  \__,_|_|  \__,_|

Processing 'hello.tex' (size: 86 bytes, last modified: 05/03/2018
07:28:30), please wait.

(PDFLaTeX) PDFLaTeX engine .............................. SUCCESS

Total: 0.73 seconds
\end{codebox}
\end{frame}

\section{Sage}
\subsection{Un programa Sage con variables}

\begin{frame}
\frametitle{\insertsection}
\begin{block}{\insertsubsection}
\begin{itemize}
\item Es un sistema computarizado algebraico.
\item Utiliza el lenguaje de propósito general Python.
\item Creado por el matemático de la Universidad de Washington, William Stein, en el año 2005.
\item Sage reutiliza software libre existentes, algunos de ellos son GAP, PARI-GP, Maxima y Singular.
\item Está escrito completamente en Python.
\end{itemize}
\end{block}
\end{frame}

\begin{frame}[fragile]
\frametitle{\insertsection}
\begin{block}{\insertsubsection}
\begin{sageblock}
f(x) = exp(x) * sin(2*x)
\end{sageblock}
The second derivative of $f$ is

\[
\frac{\mathrm{d}^{2}}{\mathrm{d}x^{2}} \sage{f(x)} =
\sage{diff(f, x, 2)(x)}.
\]

Here's a plot of $f$ from $-1$ to $1$:
\end{block}

\begin{sagesilent}
plt  = plot(f, -1, 1)
plt.save("MyPic.pdf")
\end{sagesilent}

\begin{figure}
\centering
\includegraphics[height=3cm]{MyPic.pdf}
\end{figure}

\end{frame}

\begin{frame}
\begin{block}{Modelo matemático}
Nuestro primer ejemplo se refiere a la programación de un modelo matemático que predice la posición de una pelota lanzada al aire. De la segunda ley de Newton, y asumiendo que la resistencia del aire es insignificante, se puede derivar un modelo matemático que predice la posición $y$ de la pelota en el tiempo $t$.
\end{block}

La declaración $v_0$ = $5$ se llama \emph{asignación}
\end{frame}
%%!TEX root = presentazionelancia.tex
\section{Objetivos}

\begin{frame}
\frametitle{\insertsection}
 	Queremos formar un grupo humano en la UNI alrededor de \LaTeX{} tal como lo hace el Grupo de Usuarios \TeX{} de la Universidad de Tokio.
 	
 	\url{http://ut-tex.org/index.php}
 	\begin{center}
 	\includegraphics[width=\textwidth]{figs/tokio}		
 	\end{center}
\end{frame}

\begin{frame}[fragile]
\frametitle{¿Cómo lo podemos lograr?}
\onslide<1->
\begin{itemize}
	\item Estudiar los libros de Herbert Voss, vicepresidente del Grupo de Usuarios \TeX{} Germano hablantes.
	\item Conocer herramientas amigas de \LaTeX{} como Sage, arara, PSTricks, knitr y más tecnologías. 
	\item Con el uso de un sistema de control de versiones descentralizado como \texttt{git}
	\item Herramientas de comunicación como \texttt{gitter} o \texttt{Hangouts}.
\end{itemize}
\onslide<2->
Conocimientos previos:
\begin{itemize}
	\item Comandos básicos en \texttt{UNIX}, \LaTeX{} y programación.
\end{itemize}
\end{frame}


%\begin{frame}[fragile]
%\frametitle{Associations}
%    \begin{itemize}
%	\item Typed directed edges between objects (type is denoted by \verb!atype!)
%	\item Identified by source object \verb!id1!, \verb!atype! and destination object \verb!id2!
%	\item Contains data in the form of key-value pairs.
%	\item Contains a 32-bit \verb!time! field.
%	\item Models actions that happen at most once or records state transition (e.g. like)
%	\item Often inverse association is also meaningful (eg like and liked by).
%\end{itemize}
%\end{frame}

%\begin{frame}
%\frametitle{Associations API}
%\begin{itemize}
%\item Add new
%\item Delete
%\item Change type
%\end{itemize}
%Also inverse association is created or modified automatically
%\end{frame}

%\begin{frame}[fragile]
%\frametitle{Querying TAO}
%TAO's associations queries are organized around \emph{associations~ list}
%
%\begin{itemize}
%\item \verb!assoc_get(id1,atype, id2set, high?, low?)!
%\item \verb!assoc_count(id1,atype)!
%\item \verb!assoc_range(id1, atype, pos, limit)!
%\item \verb!assoc_time_range(id1,atype, high, low, limit)!
%\end{itemize}
%
%Query results are bounded to 6000 results
%\end{frame}




%\input{architechture}
%\input{implementation}
%\input{consistency}
%%!TEX root = presentazionelancia.tex
\section{Workload \& Performance}
%\begin{frame}[c]\frametitle{Workload}
%\centering
%\includegraphics[width=\textwidth]{figs/table3.pdf}
%
%Frequencies for client request
%
%
%
%\end{frame}

%\begin{frame}[c]\frametitle{Disponibilidad}
%Bajo carga de trabajo real, durante un período de 90 días, la \textbf{fracción de consultas TAO fallidas} es:
%\begin{center}
%	\huge $4.9 \times 10^{-6}$
%\end{center}
%
%\end{frame}
%\begin{frame}[c]\frametitle{Followers Capacity}
%	\centering
%    \includegraphics[width=\textwidth]{figs/followercapacity.pdf}
%\end{frame}

%\begin{frame}[c]\frametitle{Hit Rates and read latency}
%\centering
%\includegraphics[width=\textwidth]{figs/table8.pdf} 
%
%\end{frame}

%\begin{frame}[c]\frametitle{Write Latency}
%   	\includegraphics[width=\textwidth]{figs/writes.pdf}
%\end{frame}


\begin{frame}[c]\frametitle{Resumiendo}
\begin{description}
	\item[Instalación] \hfill\\
	\begin{itemize}
		\item \TeX{} Live 2018
		\item OS GNU/Linux
		\item Sage 8.4
	\end{itemize}
	\item[Estudio]\hfill\\
	\begin{itemize}
		\item Teoría de los libros
		\item Práctica con ejercicios de ejemplos
	\end{itemize}
	\item[Comunicación y documentación]\hfill\\
	\begin{itemize}
		\item A través de Gitter o Hangout + GitHub.
	\end{itemize}
\end{description}


\end{frame}

\setbeamercolor{background canvas}{bg=matbluedark}
\setbeamercolor{normal text}{fg=white}
\begin{frame}[plain, b]
\centering
\huge \textcolor{white}{¡Gracias por venir!}
\normalsize
\begin{center}
	\includegraphics[width=0.5\textwidth]{DimensionR}\\
\end{center}
\vspace*{\fill}

 \begin{beamercolorbox}[wd=\paperwidth]{section in head/foot}
 \centering\large
\LaTeX{} \& friends -- con el soporte de Dimension R
\vskip10pt
\end{beamercolorbox}
 \end{frame}

\end{document}
